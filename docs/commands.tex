%% NEW COMMANDS
\newcommand{\Deriv}[1]{\mathcal{D}_{#1}}
\newcommand{\Incl}{\texttt{incl}}
\newcommand{\Intro}[1]{\mathbf{I}_{#1}}
\newcommand{\Elim}[1]{\mathbf{E}_{#1}}
\newcommand{\Assert}[1]{\mathrel{\color{blue}{+}}{#1}}
\newcommand{\Deny}[1]{\mathrel{\color{red}{-}}{#1}}
\newcommand{\AssertRule}[1]{{+}{#1}}
\newcommand{\DenyRule}[1]{{-}{#1}}
\newcommand{\Set}[1]{\texttt{\color{Green}{Set }}{#1}}
\newcommand{\Label}[1]{{\![#1]}}
\newcommand{\AbsCompl}[1]{\overline{#1}}
\newcommand{\AbsComplRule}{^{\mbox{---}\,}}
\newcommand{\Tuple}[1]{\left(#1\right)}
\newcommand{\Subst}[3]{#1[#2\mapsto#3]}
\newcommand{\Spec}{\mathbf{spec}\,}
\newcommand{\SetB}[4]{\{#1\in#2\mid\Subst{#3}{#4}{#1}\}}
\newcommand{\Opp}[1]{\widetilde{\smash{#1}}}
\newcommand{\OppN}[1]{\widetilde{#1}}
\newcommand{\Place}{\bigcirc}
\newcommand{\Universal}{\mathcal U}
\newcommand{\Powerset}{\wp\,}
\newcommand{\eq}{\approx}

\usepackage{accents}
\newcommand{\dbtilde}[1]{\accentset{\approx}{#1}}
% another one
\usepackage[extra]{tipa}
\newcommand{\dbltilde}[1]{\kern-1pt$#1$}

% Building a symbol for the Disjointness relation
%\newcommand{\Disjoint}{\mathrel{\rbrbrak\!\lbrbrak}} % uses stix
%\newcommand{\Disjoint}{\mathrel{\rangle\!\langle}}
\makeatletter
\DeclareFontEncoding{LS2}{}{\@noaccents}
\makeatother
\DeclareFontSubstitution{LS2}{stix}{m}{n}
%
\DeclareSymbolFont{largesymbolsstix}{LS2}{stixex}{m}{n}
%
%\DeclareMathDelimiter{\lbrbrak}{\mathopen}{largesymbolsstix}{"EE}{largesymbolsstix}{"14}
%\DeclareMathDelimiter{\rbrbrak}{\mathclose}{largesymbolsstix}{"EF}{largesymbolsstix}{"15}
\DeclareMathDelimiter{\lbrbrak}{\mathrel}{largesymbolsstix}{"EE}{largesymbolsstix}{"14}
\DeclareMathDelimiter{\rbrbrak}{\mathrel}{largesymbolsstix}{"EF}{largesymbolsstix}{"15}
\newcommand{\Disjoint}{\rbrbrak\lbrbrak}